\section{Introduction}
\subsection{Research Question}
One of the largest challenges facing our planet today is that of global warming. As our energy demands increase, so will our energy production. As long as we are not prioritizing emissions free fuel sources such as solar, wind, and nuclear energy, we will see an increase in greenhouse gas emissions as we increase our energy output. There is a large body of evidence that links greenhouse gas emissions to an increase in temperature. In this work, we will explore this relationship. Our research question is:

\begin{quote}
    Can we build a model to determine temperature of a region based on greenhouse gas emissions?
\end{quote}
We will be exploring states which emit the most greenhouse gases and which areas of the United States are hotter. Additionally, we will explore the relationship between Latitude and temperature, as there is evidence that the poles are warming faster than other parts of the world. 

We will be investigating this using the data provided by the Global Historical Climatology Network (GHCN), which contains the average daily temperature recorded at monitoring stations around the world. We will also be using the data provided by the Environmental Protection Agency (EPA) through their Greenhouse Gas Reporting System. 